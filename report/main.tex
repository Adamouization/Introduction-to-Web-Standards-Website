\documentclass[letterpaper,12pt]{article}
\usepackage{tabularx} % extra features for tabular environment
\usepackage{amsmath}  % improve math presentation
\usepackage{graphicx} % takes care of graphic including machinery
\usepackage[margin=0.95in,letterpaper]{geometry} % decreases margins
\usepackage{cite} % takes care of citations
\usepackage[titletoc,title]{appendix} % takes care of appendices
\usepackage{listings} % code representation
\usepackage{pdflscape}
\usepackage{csquotes} % for quoting existing work
\usepackage{color} % defines colors for code listings
\usepackage{comment} % allows for block of comments
\usepackage{gensymb} % degree symbol
\usepackage[final]{hyperref} % adds hyper links inside the generated pdf file

% style code listings
\definecolor{codegreen}{rgb}{0,0.6,0}
\definecolor{codegray}{rgb}{0.5,0.5,0.5}
\definecolor{backcolour}{rgb}{0.95,0.95,0.92}
\lstdefinestyle{mystyle}{
    backgroundcolor=\color{backcolour},   
    commentstyle=\color{codegreen},
    keywordstyle=\color{blue},
    numberstyle=\tiny\color{codegray},
    basicstyle=\footnotesize,
    breakatwhitespace=false,         
    breaklines=true,                 
    captionpos=b,                    
    keepspaces=true,                 
    numbers=left,                    
    numbersep=5pt,                  
    showspaces=false,                
    showstringspaces=false,
    showtabs=false,                  
    tabsize=4
}
\lstset{style=mystyle}

\begin{document}

\title{IS5103 Web Technologies\\Assignment 1 Report}
\author{Student ID: 150014151}
\date{30th October, 2019}
\maketitle
\newpage

\tableofcontents
\newpage

\section{Prior Planning and Research}

The first step consisted in deciding which area to present. The ``Role of Web Standards'' was elected. The next step was to start researching content to present and come up with a plan. To do so, the entire content, including titles, paragraphs, citations, figures and references, was typed out in a Word document before being transformed into a web page. The references used to research the content can be found directly on the website: \url{https://agj6.host.cs.st-andrews.ac.uk/references.html}

\section{Aims of the Website}

The aim of this website is to instruct the reader about the objective of web standards, how they evolved to regulate compatibility across of the World Wide Web, what are web standards organisations and who are the main ones, how new web standards are developed, and an overview of the most popular web standards.

\section{Target Audience}

The target audience for this website includes anyone who is interested in discovering how the web pages they visit on a daily basis are regulated, no matter on the level of knowledge in web development.

\section{Design and Content}

Content and markup:

write content in word
create an empty html file with populate head metadata and empty body
then copy and paste all text into body
add headings to separate sections and subsections (h1 for title, h2 for sections and h3 for subsections)
surround paragraphs with <p> tags
make unordered lists for items that required it
add images that were present in the word document
add references to elements within the file (link references and link sections when they are mentioned) 

followed guide to make a responsive navbar on small screens https://www.w3schools.com/howto/howto_js_topnav_responsive.asp

compress images for faster loading https://compressjpeg.com/


style:

write general css such as body
add font OpenSans-Regular for readability: "high readability and friendly appearance. Open Sans has excellent legibility and its letterforms are incredibly strong with the very extensive font library, this font is a very strong substitute for default sans serif fonts." https://www.flyinghippo.com/blog/10-web-fonts-write-home/

Used Chrome's developer tools to modify the css on the go and determine the best values for the site e.g. fonts


minimise css for quicker page loading https://cssminifier.com/

focus noticeable for disabled people

\section{Accessibility}

Accessibility was 

colour scheme https://visme.co/blog/website-color-schemes/

mention contrast level using chrome developer tool

keyboard accessibility

any link in focus is surrounded by a thick bright-red border

\section{Testing and Evaluation}

todo

\section{Final Conclusions}

todo


% -------------------------------------- APPENDICES ------------------------------------------ 
\begin{appendices}

\clearpage
\section{Evidence of Prior Planning}
todo

\clearpage
\section{User evaluation questionnaires}
todo

\clearpage
\section{Accessibility Results}
todo

\clearpage
\section{Validation URL's and Reports}
todo

\clearpage
\section{Browser Functionality and Compability}
todo

\newpage
\bibliographystyle{plain}
\bibliography{bibliography}

\end{appendices}
\end{document}